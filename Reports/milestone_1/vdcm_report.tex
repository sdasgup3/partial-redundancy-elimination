\begin{flushleft}
\textbf{\Large{Value-Number driven code motion}}
\end{flushleft}
We implement an iterative bit vector data flow algorithm for PRE. We initially
implemented the flow equations from the Lazy Code Motion paper. This set
included a total of 13 bit vectors for each basic block - 2 for block local
properties ANTLOC and TRANSP, and 11 for global properties. These equations,
           however, could only be applied to single instruction basic blocks.
           We therefore, derived a new set of equations which are motived by
           later work\cite{Knoop:1994:OCM:183432.183443} of the same authors.
           This set of equations apply to maximal basic blocks and
           entails a total of 19 bit vectors for each basic block in our
           current implementation - 3 for block local properties ANTLOC,
           TRANSP, XCOMP and 16 for global properties.  We include the
           equations in appendix A and B. In appendix C, we provide our generalized
           data flow framework, and show how each PRE equation maps to the
           framework. We call the algorithm value-number driven because each
           slot in each of the bit vectors is reserved for a particular value
           number rather than a particular expression. Also, we make the
           observation that a large number of expressions in the program only
           occur once, and are not useful for PRE. Hence to further optimize
           for space and time, we only give bit vector slots to value numbers
           which have more than one expression linked to them.

\begin{flushleft}
\textbf{\large{Local CSE}}
\end{flushleft}
For our data flow equations to work correctly, a local CSE pass has to be run
on each basic block. Basically, this path removes the redundancies inside
straight line basic block code and sanitizes it for the iterative bit vector
algorithm. This idea is borrowed from \cite{Knoop:1994:OCM:183432.183443}. We perform this step before calling
our data flow framework. 

\vspace{1 mm}
\begin{flushleft}
\textbf{\Large{Status}}
\end{flushleft}
We summarize accomplished work in this section. We completed the value numbering pass and included some 
optimizations as part of it. We then derived PRE equations for maximal basic block CFG and molded them 
to work with value numbers rather than lexical names. This was followed by construction of a generalized 
data flow framework which could incorporate PRE equations. More specifically, we define a single function 
which can be called with dataflow equation specific parameters. We also wrote a pass to perform local CSE on each basic block.
We have tested our implementation of small code segments with good amount of success. We are able to identify 
the placements in the CFG which are computationally optimal as well as lifetime optimal. As an example, we present the CFG for a 
fairly involved program we wrote using goto statements(Figure \ref{fig:2}). The figure on the right illustrates the placement
suggestions provided by our PRE pass.

\begin{flushleft}
\textbf{\Large{Ongoing Work and Future Tasks}}
\end{flushleft}
Having identified the INSERT points (where computation should be added) and REPLACE points (from where computation should be removed), 
we are now working on doing the actual IR modifications. Once completed we would begin with testing on fairly large codes for 1) 
correctness and 2) performance improvements. This would be followed by a comparison with the LLVM GVN-PRE pass. Time permitting, 
we would like to provide support for eliminating redundancies where two expressions are lexically different and have different value numbers 
(mentioned in section \textbf{Types Of Redundancies}).

           
